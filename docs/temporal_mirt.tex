\documentclass{article} 
\newcommand{\set}[1]{\lbrace #1 \rbrace}
\newcommand{\setc}[2]{\lbrace #1 \mid #2 \rbrace}
\newcommand{\vv}[1]{{\mathbf{#1}}}
\newcommand{\dd}{{\mathrm{d}}}
\newcommand{\pd}[2]{\frac{\partial #1}{\partial #2}}
\newcommand{\pdn}[3]{\frac{\partial^#1 #2}{\partial #3^#1}}
\newcommand{\od}[2]{\frac{\dd #1}{\dd #2}}
\newcommand{\odn}[3]{\frac{\dd^#1 #2}{\dd #3^#1}}
\newcommand{\avg}[1]{\left< #1 \right>}
\newcommand{\mb}{\mathbf}
\newcommand{\mc}{\mathcal}
\newcommand{\argmin}{\operatornamewithlimits{argmin}}

\usepackage{times}
\usepackage{amsmath,amssymb}
\usepackage{graphicx}

\title{
Temporal multidimensional item response theory
}	

\author{
Jascha Sohl-Dickstein
}

\newcommand{\fix}{\marginpar{FIX}}
\newcommand{\new}{\marginpar{NEW}} 

\begin{document}

\maketitle

$a_{im}^t$ gives the $m$th ability for student $i$ at timestep $t$.  $x_{ij}^t \in \{0,1\}$ is the correctness value of exercise $j$ at timestep $t$ for student $i$.  The conditional probability of getting an exercise correct is
\begin{align}
p\left( x_{ij}^t = 1 | \mb a_i^t \right) & = \sigma\left( \mb W_j \mb a_i^t + b_j \right),
\label{eq cond x}
\end{align}
where $W_{jm}$ is the coupling parameter between ability $m$ and exercise $j$, and $b_j$ is the bias associated with exercise $j$.  
$r_i^t$ is the resource student $i$ is exposed to at timestep $t$, where a resource might be an exercise, a video, or other content.  An exercise answered correctly and the same exercise answered incorrectly are treated as two separate resources.

The prior distribution over the initial ability state is a unit norm Gaussian,
\begin{align}
p\left( \mb a_i^1 \right) & = \mc N \left( \mb 0, \mb I\right).
\end{align}
The conditional distribution over the abilities at the next timestep given the current timestep is also given by a Gaussian,
\begin{align}
p\left(  \mb a_i^{t+1} | \mb a_i^t, r_i^t \right) & = \mc N \left( \left(\mb I + \mb \Phi_{r_i^t}\right) \mb a_i^t + \theta_{r_i^t}, \mb \Sigma_{r_i^t} \right)
,
\label{eq cond a}
\end{align}
where the matrix $\mb \Phi_{r_i^t}$ and the bias $\theta_{r_i^t}$ are defined for each resource, and determine the mean prediction for the abilities vector at the next time step.  The covariance matrix $\mb \Sigma_{r_i^t}$, also defined for each resource, determines the uncertainty in the abilities vector at the next time step.

The joint energy function is
\begin{align}
\nonumber
E = & 
\frac{1}{2} \left(\mb a_i^1\right)^T \mb a_i^1 \\ \nonumber
& + \frac{1}{2}\sum_{t=1}^{T-1} 
	\left[\mb a_i^{t+1} -   \left(\mb I + \mb \Phi_{r_i^t}\right) \mb a_i^t \right]^T
		\mb \Sigma_{r_i^t}^{-1}
	\left[\mb a_i^{t+1} -   \left(\mb I + \mb \Phi_{r_i^t}\right) \mb a_i^t \right]  \\ \nonumber
& +\sum_{t=1}^{T-1} 
	\log \det\left( \mb \Sigma_{r_i^t} \right) \\ 
& - \sum_{x_{ij}^t} \log \sigma\left( \mb W_j \mb a_i^t + b_j \right)
.
\label{eq energy}
\end{align}
It has gradients
\begin{align}
\pd{E}{W_{km}} = 
& - \sum_{x_{ij}^t} \left( 1 - \sigma\left( \mb W_j \mb a_i^t + b_j \right) \right) a_{im}^t \delta_{kj}
\label{eq grad W}
, \\
\pd{E}{\mb \Phi_{r_i^t}} = & 
\sum_{t=1}^{T-1}
	\mb \Sigma_{r_i^t}^{-1}
	\left[\mb a_i^{t+1} -   \left(\mb I + \mb \Phi_{r_i^t}\right) \mb a_i^t \right]
	\left(
		\mb a_i^t
	\right)^T
\label{eq grad phi}
, \\
\pd{E}{\mb \Sigma_{r_i^t}^{-1}} = & 
\frac{1}{2} \sum_{t=1}^{T-1}
	\left[\mb a_i^{t+1} -   \left(\mb I + \mb \Phi_{r_i^t}\right) \mb a_i^t \right]
	\left[\mb a_i^{t+1} -   \left(\mb I + \mb \Phi_{r_i^t}\right) \mb a_i^t \right]^T
\nonumber \\
& + \sum_{t=1}^{T-1}
	\left(\mb \Sigma_{r_i^t}\right)^T
\label{eq grad sigma}
\end{align}
The bias gradients can be computed by adding a ``bias unit" to the abilities vector.

\end{document}
